\documentclass[12pt,letterpaper]{article}
\usepackage[utf8]{inputenc}
\usepackage[spanish]{babel}
\usepackage{amsmath}
\usepackage{amsfonts}
\usepackage{amssymb}
\usepackage{makeidx}
\usepackage{graphicx}
\usepackage{lmodern}
\usepackage{kpfonts}
\usepackage{fourier}
\usepackage{color}
\usepackage{listings}
\usepackage{hyperref}
\usepackage{multicol}
\usepackage{multirow, array}
\usepackage{enumerate}
\usepackage[left=2.5cm,right=2cm,top=2cm,bottom=2cm]{geometry}
\usepackage{float}
\usepackage{fancyhdr}
\usepackage{quotchap}
\usepackage{tikz}
\usepackage[hidelinks]{hyperref}
\usepackage{chemfig}
\usepackage{pgfplots}
\usepackage{xcolor}	
\usepackage{titlesec}		
%% =============================================================================
%%  OTRA COSA QUE NO SEAN PAKETES
%% =============================================================================
\pagestyle{fancy}
\setcounter{secnumdepth}{3}
\setcounter{tocdepth}{4}
\fancyhf{}
\rhead{I5839 D-03}
\lhead{Laboratorio de reactores(2020B)}
\rfoot{\thepage}
%\renewcommand\thesection{\Roman{section}}	% Numeración romana en las secciones
%\renewcommand\thesubsection{\Roman{subsection}}		% Numeración romana en las subsecciones
\titlespacing*{\section}{0pt}{2.5mm}{0mm}	% Espaciado del título {espacio izquierdo}{arriba del título}{abajo del título}
\titleformat{\section}[block]{\large\scshape\centering}{\thesection.}{1em}{}
\titleformat{\subsection}[block]{\large}{\thesubsection.}{1em}{}
%\newcommand{\colorhrule}[3]{\begingroup\color{#1}\rule{#2}{#3}\endgroup}

\setlength{\intextsep}{1mm} % Distancia superior e inferior en objetos flotantes
\setlength{\columnsep}{5mm} % Separación entre columnas del documento
\spanishdecimal{.}
%% =============================================================================
%%    INICIO DEL DOCUMENTO
%% =============================================================================
\begin{document}

\renewcommand{\tablename}{Tabla}
\thispagestyle{empty}
%======================
%       PORTADA
%======================
\sloppy     % Evita que las palabras se corten al saltar de línea.
\begin{center}
  \begin{tabular}{cc}

\multirow{2}{3.5cm}{\includegraphics[width=3cm]{Figuras/udgn.eps}}	& \huge{\textsc{\textbf{Universidad de Guadalajara}}}\\
 & \scriptsize{\textsc{CENTRO UNIVERSITARIO DE CIENCIAS EXACTAS E INGENIERÍAS}}\\[5mm]
 & \Large{\textsf{\textbf{Práctica 4. Distribución de tiempos de residencia
}}}\\
 & \\ \vspace{5mm}
 & \small{\textsf{Alejandro Leviatán Gallifrey  [213521903]}}\\
 & \small{\textsc{Ingeniería Química $|$  - Laboratorio de reactores Químicos}}\\
 & \today\\
 & \small{\textit{Profersor:}}  \textbf{\small{Alejandro Nava Tellez }}\\
  \end{tabular}
\end{center}

%\begin{center}
 % \colorhrule{negro}{16.5cm}{1.2pt}
%\end{center}

\rule{\linewidth}{0.75mm}

\tableofcontents
\section{Objetivos}

\begin{figure}[H]
	\centering
	\includegraphics[scale=1]{Figuras/Curva_Calibracion.pdf}
	\caption{Ejemplo}
	\label{fig:Curva_calibracion}
	
\end{figure}



\end{document}